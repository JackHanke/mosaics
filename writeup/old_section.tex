

We first re-prove the enumeration of $k_{n,m}$ from Theorem \ref{thm:Oh2014} using our general method for enumerating mosaic systems, and show it immediately gives an an analogous result for $p_{n,m}$. We state the enumeration of $p_{n,m}$ for $m,n \geq 2$ in Theorem \ref{thm: main theorem}. First we define $A(m) \in \mathbb{Z}^{2^{m-1} \times 2^{m-1}}$ for integers $m \geq 2$, which we use to enumerate $p_{m,n}$. To be clear, throughout the paper we index the rows and columns of matrices starting at $0$.

\begin{defn}
\label{defn: A}

Let $A(2) = \begin{bmatrix}
1 & 1 \\
1 & 1
\end{bmatrix}
$. We recursively define $A(k+1)$ given $A(k)$. Begin by writing
$
A(k) = \begin{bmatrix}
A_{0,0} & A_{0,1} \\
A_{1,0} & A_{1,1}
\end{bmatrix}
$, where the block matrices $A_{i,j}$ are square block matrices of size $2^{k-2} \times 2^{k-2}$. We then define

$$
A(k+1) = \begin{bmatrix}
A_{0,0} & A_{0,0} & A_{0,1} & A_{0,1} \\
A_{0,0} & A_{0,0} & 0A_{0,1} & A_{0,1} \\
A_{1,0} & 0A_{1,0} & A_{1,1} & A_{1,1} \\
A_{1,0} & A_{1,0} & A_{1,1} & A_{1,1} \\
\end{bmatrix},
$$

Construct $A(m)$ by starting with $k=2$ and recursing until $k=m$. 

\end{defn}

\begin{thm}
\label{thm: main theorem}    
The number of polygon mosaics $p_{m,n}$ is the $(0,0)$ entry of $A(m)^n$.
\end{thm}

We then introduce messy knot mosaics—a variant of knot mosaics— in Section \ref{section:messy mosaics} and enumerate them.

\section{Proof of Theorem \ref{thm: main theorem}}

\begin{proof}

% In doing so, we avoid the use of the so-called \textit{twofold rule} in \cite{Oh2014} to 

For a given mosaic, label the vertices of the tiles as follows. If the vertex is surrounded by an even number of polygons, label it $0$. If the vertex is surrounded by an odd number of polygons, label it $1$. To make a \textit{vertex labeling} we also remove the dotted lines from all tiles in the mosaic. Using the mosaic from Figure \ref{fig:example knot mosaic}, we show both the labeling of the vertices, and then the removal of the dotted lines.

\begin{center}
    \begin{tikzpicture}
        % row1
        \cell{0}{0}{1}{1}
        \cellC{1}{0}{2}{1}
        \cellD{2}{0}{3}{1}
        \cellC{3}{0}{4}{1}
        \cellE{4}{0}{5}{1}
        \cellE{5}{0}{6}{1}
        \cellD{6}{0}{7}{1}
        % row2
        \cell{0}{1}{1}{2}
        \cellB{1}{1}{2}{2}
        \cellI{2}{1}{3}{2}
        \cellA{3}{1}{4}{2}
        \cellC{4}{1}{5}{2}
        \cellD{5}{1}{6}{2}
        \cellF{6}{1}{7}{2}
        % row3
        \cell{0}{2}{1}{3}
        \cell{1}{2}{2}{3}
        \cellF{2}{2}{3}{3}
        \cell{3}{2}{4}{3}
        \cellB{4}{2}{5}{3}
        \cellA{5}{2}{6}{3}
        \cellF{6}{2}{7}{3}
        % row4
        \cellC{0}{3}{1}{4}
        \cellD{1}{3}{2}{4}
        \cellB{2}{3}{3}{4}
        \cellE{3}{3}{4}{4}
        \cellD{4}{3}{5}{4}
        \cell{5}{3}{6}{4}
        \cellF{6}{3}{7}{4}
        % row5
        \cellB{0}{4}{1}{5}
        \cellA{1}{4}{2}{5}
        \cell{2}{4}{3}{5}
        \cell{3}{4}{4}{5}
        \cellB{4}{4}{5}{5}
        \cellE{5}{4}{6}{5}
        \cellA{6}{4}{7}{5}
        % label for row1
        \( \lablvertex{0}{0}{$0$} \)
        \( \lablvertex{1}{0}{$0$} \)
        \( \lablvertex{2}{0}{$0$} \)
        \( \lablvertex{3}{0}{$0$} \)
        \( \lablvertex{4}{0}{$0$} \)
        \( \lablvertex{5}{0}{$0$} \)
        \( \lablvertex{6}{0}{$0$} \)
        \( \lablvertex{7}{0}{$0$} \)
        % label for row1
        \( \lablvertex{0}{1}{$0$} \)
        \( \lablvertex{1}{1}{$0$} \)
        \( \lablvertex{2}{1}{$1$} \)
        \( \lablvertex{3}{1}{$0$} \)
        \( \lablvertex{4}{1}{$1$} \)
        \( \lablvertex{5}{1}{$1$} \)
        \( \lablvertex{6}{1}{$1$} \)
        \( \lablvertex{7}{1}{$0$} \)
        % label for row1
        \( \lablvertex{0}{2}{$0$} \)
        \( \lablvertex{1}{2}{$0$} \)
        \( \lablvertex{2}{2}{$0$} \)
        \( \lablvertex{3}{2}{$1$} \)
        \( \lablvertex{4}{2}{$1$} \)
        \( \lablvertex{5}{2}{$0$} \)
        \( \lablvertex{6}{2}{$1$} \)
        \( \lablvertex{7}{2}{$0$} \)
        % label for row1
        \( \lablvertex{0}{3}{$0$} \)
        \( \lablvertex{1}{3}{$0$} \)
        \( \lablvertex{2}{3}{$0$} \)
        \( \lablvertex{3}{3}{$1$} \)
        \( \lablvertex{4}{3}{$1$} \)
        \( \lablvertex{5}{3}{$1$} \)
        \( \lablvertex{6}{3}{$1$} \)
        \( \lablvertex{7}{3}{$0$} \)
        % label for row1
        \( \lablvertex{0}{4}{$0$} \)
        \( \lablvertex{1}{4}{$1$} \)
        \( \lablvertex{2}{4}{$0$} \)
        \( \lablvertex{3}{4}{$0$} \)
        \( \lablvertex{4}{4}{$0$} \)
        \( \lablvertex{5}{4}{$1$} \)
        \( \lablvertex{6}{4}{$1$} \)
        \( \lablvertex{7}{4}{$0$} \)
        % label for row1
        \( \lablvertex{0}{5}{$0$} \)
        \( \lablvertex{1}{5}{$0$} \)
        \( \lablvertex{2}{5}{$0$} \)
        \( \lablvertex{3}{5}{$0$} \)
        \( \lablvertex{4}{5}{$0$} \)
        \( \lablvertex{5}{5}{$0$} \)
        \( \lablvertex{6}{5}{$0$} \)
        \( \lablvertex{7}{5}{$0$} \)
        % arrow
        \( \lablnode{7.5}{2.5}{$\pmb{\to}$} \)

        % row1
        \cell{8}{0}{9}{1}
        \cell{9}{0}{10}{1}
        \cell{10}{0}{11}{1}
        \cell{11}{0}{12}{1}
        \cell{12}{0}{13}{1}
        \cell{13}{0}{14}{1}
        \cell{14}{0}{15}{1}
        % row2
        \cell{8}{1}{9}{2}
        \cell{9}{1}{10}{2}
        \cell{10}{1}{11}{2}
        \cell{11}{1}{12}{2}
        \cell{12}{1}{13}{2}
        \cell{13}{1}{14}{2}
        \cell{14}{1}{15}{2}
        % row3
        \cell{8}{2}{9}{3}
        \cell{9}{2}{10}{3}
        \cell{10}{2}{11}{3}
        \cell{11}{2}{12}{3}
        \cell{12}{2}{13}{3}
        \cell{13}{2}{14}{3}
        \cell{14}{2}{15}{3}
        % row4
        \cell{8}{3}{9}{4}
        \cell{9}{3}{10}{4}
        \cell{10}{3}{11}{4}
        \cell{11}{3}{12}{4}
        \cell{12}{3}{13}{4}
        \cell{13}{3}{14}{4}
        \cell{14}{3}{15}{4}
        % row5
        \cell{8}{4}{9}{5}
        \cell{9}{4}{10}{5}
        \cell{10}{4}{11}{5}
        \cell{11}{4}{12}{5}
        \cell{12}{4}{13}{5}
        \cell{13}{4}{14}{5}
        \cell{14}{4}{15}{5}

        % label for row1
        \( \lablvertex{8}{0}{$0$} \)
        \( \lablvertex{9}{0}{$0$} \)
        \( \lablvertex{10}{0}{$0$} \)
        \( \lablvertex{11}{0}{$0$} \)
        \( \lablvertex{12}{0}{$0$} \)
        \( \lablvertex{13}{0}{$0$} \)
        \( \lablvertex{14}{0}{$0$} \)
        \( \lablvertex{15}{0}{$0$} \)
        % label for row1
        \( \lablvertex{8}{1}{$0$} \)
        \( \lablvertex{9}{1}{$0$} \)
        \( \lablvertex{10}{1}{$1$} \)
        \( \lablvertex{11}{1}{$0$} \)
        \( \lablvertex{12}{1}{$1$} \)
        \( \lablvertex{13}{1}{$1$} \)
        \( \lablvertex{14}{1}{$1$} \)
        \( \lablvertex{15}{1}{$0$} \)
        % label for row1
        \( \lablvertex{8}{2}{$0$} \)
        \( \lablvertex{9}{2}{$0$} \)
        \( \lablvertex{10}{2}{$0$} \)
        \( \lablvertex{11}{2}{$1$} \)
        \( \lablvertex{12}{2}{$1$} \)
        \( \lablvertex{13}{2}{$0$} \)
        \( \lablvertex{14}{2}{$1$} \)
        \( \lablvertex{15}{2}{$0$} \)
        % label for row1
        \( \lablvertex{8}{3}{$0$} \)
        \( \lablvertex{9}{3}{$0$} \)
        \( \lablvertex{10}{3}{$0$} \)
        \( \lablvertex{11}{3}{$1$} \)
        \( \lablvertex{12}{3}{$1$} \)
        \( \lablvertex{13}{3}{$1$} \)
        \( \lablvertex{14}{3}{$1$} \)
        \( \lablvertex{15}{3}{$0$} \)
        % label for row1
        \( \lablvertex{8}{4}{$0$} \)
        \( \lablvertex{9}{4}{$1$} \)
        \( \lablvertex{10}{4}{$0$} \)
        \( \lablvertex{11}{4}{$0$} \)
        \( \lablvertex{12}{4}{$0$} \)
        \( \lablvertex{13}{4}{$1$} \)
        \( \lablvertex{14}{4}{$1$} \)
        \( \lablvertex{15}{4}{$0$} \)
        % label for row1
        \( \lablvertex{8}{5}{$0$} \)
        \( \lablvertex{9}{5}{$0$} \)
        \( \lablvertex{10}{5}{$0$} \)
        \( \lablvertex{11}{5}{$0$} \)
        \( \lablvertex{12}{5}{$0$} \)
        \( \lablvertex{13}{5}{$0$} \)
        \( \lablvertex{14}{5}{$0$} \)
        \( \lablvertex{15}{5}{$0$} \)

    \end{tikzpicture}
\end{center}

The critical point is this: even with the dotted lines removed, the vertex labeling uniquely identifies the polygon mosaic. This is true even if there are polygons surrounding other polygons. This is because the vertex labeling of an individual cell, which we will call a \textit{cell labeling}, uniquely corresponds with a cell $T_i$. This is shown below.

\begin{center}
    \begin{tikzpicture}
        % row 1
        \cell{-2}{0}{-1}{1}
        \( \lablnode{-1.5}{-0.5}{$T_1$} \) 
        \cellA{0}{0}{1}{1}
        \( \lablnode{0.5}{-0.5}{$T_2$} \) 
        \cellB{2}{0}{3}{1}
        \( \lablnode{2.5}{-0.5}{$T_3$} \) 
        \cellC{4}{0}{5}{1}
        \( \lablnode{4.5}{-0.5}{$T_4$} \) 
        \cellD{6}{0}{7}{1}
        \( \lablnode{6.5}{-0.5}{$T_5$} \) 
        \cellE{8}{0}{9}{1}
        \( \lablnode{8.5}{-0.5}{$T_6$} \) 
        \cellF{10}{0}{11}{1}
        \( \lablnode{10.5}{-0.5}{$T_7$} \) 

        \( \lablvertex{-2}{0}{$1$} \)
        \( \lablvertex{-2}{1}{$1$} \)
        \( \lablvertex{-1}{0}{$1$} \)
        \( \lablvertex{-1}{1}{$1$} \)

        \( \lablvertex{0}{0}{$1$} \)
        \( \lablvertex{0}{1}{$1$} \)
        \( \lablvertex{1}{0}{$0$} \)
        \( \lablvertex{1}{1}{$1$} \)

        \( \lablvertex{2}{0}{$0$} \)
        \( \lablvertex{2}{1}{$1$} \)
        \( \lablvertex{3}{0}{$1$} \)
        \( \lablvertex{3}{1}{$1$} \)

        \( \lablvertex{4}{0}{$1$} \)
        \( \lablvertex{4}{1}{$0$} \)
        \( \lablvertex{5}{0}{$1$} \)
        \( \lablvertex{5}{1}{$1$} \)

        \( \lablvertex{6}{0}{$1$} \)
        \( \lablvertex{6}{1}{$1$} \)
        \( \lablvertex{7}{0}{$1$} \)
        \( \lablvertex{7}{1}{$0$} \)

        \( \lablvertex{8}{0}{$1$} \)
        \( \lablvertex{8}{1}{$1$} \)
        \( \lablvertex{9}{0}{$0$} \)
        \( \lablvertex{9}{1}{$0$} \)

        \( \lablvertex{10}{0}{$1$} \)
        \( \lablvertex{10}{1}{$0$} \)
        \( \lablvertex{11}{0}{$1$} \)
        \( \lablvertex{11}{1}{$0$} \)


        % row 2
        \cell{-2}{2}{-1}{3}
        \( \lablnode{-1.5}{1.5}{$T_1$} \) 
        \cellA{0}{2}{1}{3}
        \( \lablnode{0.5}{1.5}{$T_2$} \) 
        \cellB{2}{2}{3}{3}
        \( \lablnode{2.5}{1.5}{$T_3$} \) 
        \cellC{4}{2}{5}{3}
        \( \lablnode{4.5}{1.5}{$T_4$} \) 
        \cellD{6}{2}{7}{3}
        \( \lablnode{6.5}{1.5}{$T_5$} \) 
        \cellE{8}{2}{9}{3}
        \( \lablnode{8.5}{1.5}{$T_6$} \) 
        \cellF{10}{2}{11}{3}
        \( \lablnode{10.5}{1.5}{$T_7$} \) 


        \( \lablvertex{-2}{2}{$0$} \)
        \( \lablvertex{-2}{3}{$0$} \)
        \( \lablvertex{-1}{2}{$0$} \)
        \( \lablvertex{-1}{3}{$0$} \)

        \( \lablvertex{0}{2}{$0$} \)
        \( \lablvertex{0}{3}{$0$} \)
        \( \lablvertex{1}{2}{$1$} \)
        \( \lablvertex{1}{3}{$0$} \)

        \( \lablvertex{2}{2}{$1$} \)
        \( \lablvertex{2}{3}{$0$} \)
        \( \lablvertex{3}{2}{$0$} \)
        \( \lablvertex{3}{3}{$0$} \)

        \( \lablvertex{4}{2}{$0$} \)
        \( \lablvertex{4}{3}{$1$} \)
        \( \lablvertex{5}{2}{$0$} \)
        \( \lablvertex{5}{3}{$0$} \)

        \( \lablvertex{6}{2}{$0$} \)
        \( \lablvertex{6}{3}{$0$} \)
        \( \lablvertex{7}{2}{$0$} \)
        \( \lablvertex{7}{3}{$1$} \)

        \( \lablvertex{8}{2}{$0$} \)
        \( \lablvertex{8}{3}{$0$} \)
        \( \lablvertex{9}{2}{$1$} \)
        \( \lablvertex{9}{3}{$1$} \)

        \( \lablvertex{10}{2}{$0$} \)
        \( \lablvertex{10}{3}{$1$} \)
        \( \lablvertex{11}{2}{$0$} \)
        \( \lablvertex{11}{3}{$1$} \)
    \end{tikzpicture}
\end{center}

We can then enumerate $p_{m,n}$ by enumerating the number of vertex labelings that correspond to a polygon mosaic. 

Consider the collection of vertex labelings for a $m \times n$ grid of cells. To correspond with a polygon mosaic, each boundary vertex is necessarily labeled $0$, and each interior vertex is labeled $0$ or $1$. Additionally, of these $2^{(m-1)(n-1)}$ labelings, the labelings that correspond with valid polygon mosaics are ones that do not contain the following two cell labelings.

\begin{center}
    \begin{tikzpicture}
        % row1
        \cell{0}{0}{1}{1}
        \cell{2}{0}{3}{1}

        \( \lablvertex{0}{0}{$0$} \)
        \( \lablvertex{0}{1}{$1$} \)
        \( \lablvertex{1}{0}{$1$} \)
        \( \lablvertex{1}{1}{$0$} \)

        \( \lablvertex{2}{0}{$1$} \)
        \( \lablvertex{2}{1}{$0$} \)
        \( \lablvertex{3}{0}{$0$} \)
        \( \lablvertex{3}{1}{$1$} \)
        
    \end{tikzpicture}
\end{center}

These two cell labelings aren't associated with any of the cells $T_1, \dots, T_7$, and so cannot correspond with a polygon mosaic. We seek a recursive solution to enumerate the number of vertex labelings for an $m \times n$ grid with these  conditions: the labeling has a boundary of all $0$'s and does not contain the above cell labelings. 

Our solution accomplishes this by building $m \times n$ vertex labelings for fixed $m$ using vertex labelings of $m \times 1$ columns of cells. These columns have vertex labelings such that the top and bottom edge all labeled $0$. One of these columns for $m=3$ is below.

\begin{center}
    \begin{tikzpicture}
        \cell{0}{0}{1}{1}
        \cell{0}{1}{1}{2}
        \cell{0}{2}{1}{3}

        % col 1
        \( \lablvertex{0}{0}{$0$} \)
        \( \lablvertex{0}{1}{$0$} \)
        \( \lablvertex{0}{2}{$0$} \)
        \( \lablvertex{0}{3}{$0$} \)
        
        % col 2
        \( \lablvertex{1}{0}{$0$} \)
        \( \lablvertex{1}{1}{$1$} \)
        \( \lablvertex{1}{2}{$0$} \)
        \( \lablvertex{1}{3}{$0$} \)
        
        % index
        % \( \lablnode{0.5}{-1}{$(0,2)$} \)
    \end{tikzpicture}
\end{center}

It is useful to define an index for vertex labelings of these $m \times 1$ columns. To do this, consider reading a column of vertex labelings from bottom to top, ignoring the first and last $0$'s. If we interpret these sequences for the left and right column as binary numbers $b_{left}$ and $b_{right}$, the index in base ten is the pair $(b_{left}, b_{right})$. For example, for the above $m=3$ column, we have sequences $00$ for the left column and $10$ for the right column, so the index is $(0,2)$.

Next consider a column with index $(0, b_{1})$ for some $b_{1} \in [0,2^{m-1}-1]$. For reasons we will see later, let's call this the \textit{starting column}. Now consider appending a column with index $(b_{1}, b_{2})$ for some $b_{2} \in [0,2^{m-1}-1]$. For example, below is a diagram for appending our starting column $(0,2)$ with column $(2,3)$. 

\begin{center}
    \begin{tikzpicture}
        \cell{0}{0}{1}{1}
        \cell{0}{1}{1}{2}
        \cell{0}{2}{1}{3}
        % col 1
        \( \lablvertex{0}{0}{$0$} \)
        \( \lablvertex{0}{1}{$0$} \)
        \( \lablvertex{0}{2}{$0$} \)
        \( \lablvertex{0}{3}{$0$} \)
        % col 2
        \( \lablvertex{1}{0}{$0$} \)
        \( \lablvertex{1}{1}{$1$} \)
        \( \lablvertex{1}{2}{$0$} \)
        \( \lablvertex{1}{3}{$0$} \)

        % plus
        \( \lablnode{1.5}{1.5}{$\pmb{+}$} \)

        \cell{2}{0}{3}{1}
        \cell{2}{1}{3}{2}
        \cell{2}{2}{3}{3}
        % col 1
        \( \lablvertex{2}{0}{$0$} \)
        \( \lablvertex{2}{1}{$1$} \)
        \( \lablvertex{2}{2}{$0$} \)
        \( \lablvertex{2}{3}{$0$} \)
        % col 1
        \( \lablvertex{3}{0}{$0$} \)
        \( \lablvertex{3}{1}{$1$} \)
        \( \lablvertex{3}{2}{$1$} \)
        \( \lablvertex{3}{3}{$0$} \)

        \( \lablnode{3.5}{1.5}{$\pmb{\to}$} \)

        \cell{4}{0}{5}{1}
        \cell{4}{1}{5}{2}
        \cell{4}{2}{5}{3}
        \cell{5}{0}{6}{1}
        \cell{5}{1}{6}{2}
        \cell{5}{2}{6}{3}

        % col 1
        \( \lablvertex{4}{0}{$0$} \)
        \( \lablvertex{4}{1}{$0$} \)
        \( \lablvertex{4}{2}{$0$} \)
        \( \lablvertex{4}{3}{$0$} \)
        % col 2
        \( \lablvertex{5}{0}{$0$} \)
        \( \lablvertex{5}{1}{$1$} \)
        \( \lablvertex{5}{2}{$0$} \)
        \( \lablvertex{5}{3}{$0$} \)
        % col 3
        \( \lablvertex{6}{0}{$0$} \)
        \( \lablvertex{6}{1}{$1$} \)
        \( \lablvertex{6}{2}{$1$} \)
        \( \lablvertex{6}{3}{$0$} \)


    \end{tikzpicture}
\end{center}

Notice that in the above example we have created a vertex labeling for an $m \times 2$ grid of cells, in which the left-most vertex column labels are all $0$ (left index of $0$). Also note that we did not create either illegal cell labelings with column $(2,3)$. However, we could append the column $(2,1)$, which would create an illegal cell labeling. 

Finally, if we further appended a column with label $(b_2,0)$, we would have created a vertex labeling with a boundary of all $0$'s. For this reason, let's call a column with index $(b,0)$ for $b \in [0,2^{m-1}-1]$ an \textit{ending column}. This vertex labeling would correspond with $1$ polygon mosaic if the middle column had index $(2,3)$, but not if the middle column had index $(2,1)$.

This motivates the creation of a matrix $A(m)$ to every column index $(i,j)$, where $A(m)_{i,j}=1$ if the column labeling corresponds with a legal polygon mosaic, and $0$ otherwise. For example, the $A(3)$ matrix corresponds with the following labeled columns. 

\begin{center}
    \begin{tikzpicture}
        % row1
        \cell{0}{0}{1}{1}
        \cell{0}{1}{1}{2}
        \cell{0}{2}{1}{3}

        \cell{2}{0}{3}{1}
        \cell{2}{1}{3}{2}
        \cell{2}{2}{3}{3}

        \cell{4}{0}{5}{1}
        \cell{4}{1}{5}{2}
        \cell{4}{2}{5}{3}

        \cell{6}{0}{7}{1}
        \cell{6}{1}{7}{2}
        \cell{6}{2}{7}{3}
        % row2
        \cell{0}{4}{1}{5}
        \cell{0}{5}{1}{6}
        \cell{0}{6}{1}{7}

        \cell{2}{4}{3}{5}
        \cell{2}{5}{3}{6}
        \cell{2}{6}{3}{7}

        \cell{4}{4}{5}{5}
        \cell{4}{5}{5}{6}
        \cell{4}{6}{5}{7}

        \cell{6}{4}{7}{5}
        \cell{6}{5}{7}{6}
        \cell{6}{6}{7}{7}
        % row3
        \cell{0}{8}{1}{9}
        \cell{0}{9}{1}{10}
        \cell{0}{10}{1}{11}

        \cell{2}{8}{3}{9}
        \cell{2}{9}{3}{10}
        \cell{2}{10}{3}{11}

        \cell{4}{8}{5}{9}
        \cell{4}{9}{5}{10}
        \cell{4}{10}{5}{11}

        \cell{6}{8}{7}{9}
        \cell{6}{9}{7}{10}
        \cell{6}{10}{7}{11}
        % row4
        \cell{0}{12}{1}{13}
        \cell{0}{13}{1}{14}
        \cell{0}{14}{1}{15}

        \cell{2}{12}{3}{13}
        \cell{2}{13}{3}{14}
        \cell{2}{14}{3}{15}

        \cell{4}{12}{5}{13}
        \cell{4}{13}{5}{14}
        \cell{4}{14}{5}{15}

        \cell{6}{12}{7}{13}
        \cell{6}{13}{7}{14}
        \cell{6}{14}{7}{15}
        % row 1
        \( \lablvertex{0}{0}{$0$} \)
        \( \lablvertex{0}{1}{$1$} \)
        \( \lablvertex{0}{2}{$1$} \)
        \( \lablvertex{0}{3}{$0$} \)
        
        \( \lablvertex{1}{0}{$0$} \)
        \( \lablvertex{1}{1}{$0$} \)
        \( \lablvertex{1}{2}{$0$} \)
        \( \lablvertex{1}{3}{$0$} \)
        
        \( \lablvertex{2}{0}{$0$} \)
        \( \lablvertex{2}{1}{$1$} \)
        \( \lablvertex{2}{2}{$1$} \)
        \( \lablvertex{2}{3}{$0$} \)
        
        \( \lablvertex{3}{0}{$0$} \)
        \( \lablvertex{3}{1}{$0$} \)
        \( \lablvertex{3}{2}{$1$} \)
        \( \lablvertex{3}{3}{$0$} \)
        
        \( \lablvertex{4}{0}{$0$} \)
        \( \lablvertex{4}{1}{$1$} \)
        \( \lablvertex{4}{2}{$1$} \)
        \( \lablvertex{4}{3}{$0$} \)
        
        \( \lablvertex{5}{0}{$0$} \)
        \( \lablvertex{5}{1}{$1$} \)
        \( \lablvertex{5}{2}{$0$} \)
        \( \lablvertex{5}{3}{$0$} \)
        
        \( \lablvertex{6}{0}{$0$} \)
        \( \lablvertex{6}{1}{$1$} \)
        \( \lablvertex{6}{2}{$1$} \)
        \( \lablvertex{6}{3}{$0$} \)
        
        \( \lablvertex{7}{0}{$0$} \)
        \( \lablvertex{7}{1}{$1$} \)
        \( \lablvertex{7}{2}{$1$} \)
        \( \lablvertex{7}{3}{$0$} \)
        % row 2
        \( \lablvertex{0}{4}{$0$} \)
        \( \lablvertex{0}{5}{$1$} \)
        \( \lablvertex{0}{6}{$0$} \)
        \( \lablvertex{0}{7}{$0$} \)
        
        \( \lablvertex{1}{4}{$0$} \)
        \( \lablvertex{1}{5}{$0$} \)
        \( \lablvertex{1}{6}{$0$} \)
        \( \lablvertex{1}{7}{$0$} \)
        
        \( \lablvertex{2}{4}{$0$} \)
        \( \lablvertex{2}{5}{$1$} \)
        \( \lablvertex{2}{6}{$0$} \)
        \( \lablvertex{2}{7}{$0$} \)
        
        \( \lablvertex{3}{4}{$0$} \)
        \( \lablvertex{3}{5}{$0$} \)
        \( \lablvertex{3}{6}{$1$} \)
        \( \lablvertex{3}{7}{$0$} \)
        
        \( \lablvertex{4}{4}{$0$} \)
        \( \lablvertex{4}{5}{$1$} \)
        \( \lablvertex{4}{6}{$0$} \)
        \( \lablvertex{4}{7}{$0$} \)
        
        \( \lablvertex{5}{4}{$0$} \)
        \( \lablvertex{5}{5}{$1$} \)
        \( \lablvertex{5}{6}{$0$} \)
        \( \lablvertex{5}{7}{$0$} \)
        
        \( \lablvertex{6}{4}{$0$} \)
        \( \lablvertex{6}{5}{$1$} \)
        \( \lablvertex{6}{6}{$0$} \)
        \( \lablvertex{6}{7}{$0$} \)
        
        \( \lablvertex{7}{4}{$0$} \)
        \( \lablvertex{7}{5}{$1$} \)
        \( \lablvertex{7}{6}{$1$} \)
        \( \lablvertex{7}{7}{$0$} \)
        % row 3
        \( \lablvertex{0}{8}{$0$} \)
        \( \lablvertex{0}{9}{$0$} \)
        \( \lablvertex{0}{10}{$1$} \)
        \( \lablvertex{0}{11}{$0$} \)
        
        \( \lablvertex{1}{8}{$0$} \)
        \( \lablvertex{1}{9}{$0$} \)
        \( \lablvertex{1}{10}{$0$} \)
        \( \lablvertex{1}{11}{$0$} \)
        
        \( \lablvertex{2}{8}{$0$} \)
        \( \lablvertex{2}{9}{$0$} \)
        \( \lablvertex{2}{10}{$1$} \)
        \( \lablvertex{2}{11}{$0$} \)
        
        \( \lablvertex{3}{8}{$0$} \)
        \( \lablvertex{3}{9}{$0$} \)
        \( \lablvertex{3}{10}{$1$} \)
        \( \lablvertex{3}{11}{$0$} \)
        
        \( \lablvertex{4}{8}{$0$} \)
        \( \lablvertex{4}{9}{$0$} \)
        \( \lablvertex{4}{10}{$1$} \)
        \( \lablvertex{4}{11}{$0$} \)
        
        \( \lablvertex{5}{8}{$0$} \)
        \( \lablvertex{5}{9}{$1$} \)
        \( \lablvertex{5}{10}{$0$} \)
        \( \lablvertex{5}{11}{$0$} \)
        
        \( \lablvertex{6}{8}{$0$} \)
        \( \lablvertex{6}{9}{$0$} \)
        \( \lablvertex{6}{10}{$1$} \)
        \( \lablvertex{6}{11}{$0$} \)
        
        \( \lablvertex{7}{8}{$0$} \)
        \( \lablvertex{7}{9}{$1$} \)
        \( \lablvertex{7}{10}{$1$} \)
        \( \lablvertex{7}{11}{$0$} \)
        % row 4
        \( \lablvertex{0}{12}{$0$} \)
        \( \lablvertex{0}{13}{$0$} \)
        \( \lablvertex{0}{14}{$0$} \)
        \( \lablvertex{0}{15}{$0$} \)
        
        \( \lablvertex{1}{12}{$0$} \)
        \( \lablvertex{1}{13}{$0$} \)
        \( \lablvertex{1}{14}{$0$} \)
        \( \lablvertex{1}{15}{$0$} \)
        
        \( \lablvertex{2}{12}{$0$} \)
        \( \lablvertex{2}{13}{$0$} \)
        \( \lablvertex{2}{14}{$0$} \)
        \( \lablvertex{2}{15}{$0$} \)
        
        \( \lablvertex{3}{12}{$0$} \)
        \( \lablvertex{3}{13}{$0$} \)
        \( \lablvertex{3}{14}{$1$} \)
        \( \lablvertex{3}{15}{$0$} \)
        
        \( \lablvertex{4}{12}{$0$} \)
        \( \lablvertex{4}{13}{$0$} \)
        \( \lablvertex{4}{14}{$0$} \)
        \( \lablvertex{4}{15}{$0$} \)
        
        \( \lablvertex{5}{12}{$0$} \)
        \( \lablvertex{5}{13}{$1$} \)
        \( \lablvertex{5}{14}{$0$} \)
        \( \lablvertex{5}{15}{$0$} \)
        
        \( \lablvertex{6}{12}{$0$} \)
        \( \lablvertex{6}{13}{$0$} \)
        \( \lablvertex{6}{14}{$0$} \)
        \( \lablvertex{6}{15}{$0$} \)
        
        \( \lablvertex{7}{12}{$0$} \)
        \( \lablvertex{7}{13}{$1$} \)
        \( \lablvertex{7}{14}{$1$} \)
        \( \lablvertex{7}{15}{$0$} \)

        % arrow
        \( \lablnode{9}{7.5}{$\pmb{\to}$} \)
        % matrix
        \( \lablnode{12}{7.5}{$\begin{bmatrix} 1 & 1 & 1 & 1 \\ 1 & 1 & 0 & 1 \\ 1 & 0 & 1 & 1 \\ 1 & 1 & 1 & 1 \end{bmatrix}$} \)

        \( \lablnode{12}{5.5}{$A(3)$} \)

    \end{tikzpicture}
\end{center}

$A(m)$ has the property that the $0$-th row represets all starting columns, and the $0$-th column represents all ending columns. Even more importantly, notice that $A(m)^2_{i,j}$ represents the number of $m \times 2$ grids with left-most index $i$ and right-most index $j$ that correspond with a legal polygon mosaic. In general, $(A(m)^n)_{i,j}$ represents this quantity for an $m \times n$ grid of cells, and so if we know $A(m)$ for some $m$, then $(A(m)^n)_{0,0} = p_{m,n}.$

The final component of the proof is constructing $A(m)$ for any $m$. Begin by  calculating $A(2)$ by identifying the number of legal polygon mosaics that correspond with each vertex coloring index $\{(0,0),(0,1),(1,0),(1,1)\}$, like so.

\begin{center}
    \begin{tikzpicture}
        % row1
        \cell{0}{0}{1}{1}
        \cell{0}{1}{1}{2}
        
        \cell{2}{0}{3}{1}
        \cell{2}{1}{3}{2}
        
        % row2
        \cell{0}{3}{1}{4}
        \cell{0}{4}{1}{5}
        
        \cell{2}{3}{3}{4}
        \cell{2}{4}{3}{5}
        % row 1
        \( \lablvertex{0}{0}{$0$} \)
        \( \lablvertex{0}{1}{$1$} \)
        \( \lablvertex{0}{2}{$0$} \)
        
        \( \lablvertex{1}{0}{$0$} \)
        \( \lablvertex{1}{1}{$0$} \)
        \( \lablvertex{1}{2}{$0$} \)
        
        \( \lablvertex{2}{0}{$0$} \)
        \( \lablvertex{2}{1}{$1$} \)
        \( \lablvertex{2}{2}{$0$} \)
        
        \( \lablvertex{3}{0}{$0$} \)
        \( \lablvertex{3}{1}{$1$} \)
        \( \lablvertex{3}{2}{$0$} \)
        % row 2
        \( \lablvertex{0}{3}{$0$} \)
        \( \lablvertex{0}{4}{$0$} \)
        \( \lablvertex{0}{5}{$0$} \)
        
        \( \lablvertex{1}{3}{$0$} \)
        \( \lablvertex{1}{4}{$0$} \)
        \( \lablvertex{1}{5}{$0$} \)
        
        \( \lablvertex{2}{3}{$0$} \)
        \( \lablvertex{2}{4}{$0$} \)
        \( \lablvertex{2}{5}{$0$} \)
        
        \( \lablvertex{3}{3}{$0$} \)
        \( \lablvertex{3}{4}{$1$} \)
        \( \lablvertex{3}{5}{$0$} \)

        % arrow
        \( \lablnode{5}{2.5}{$\pmb{\to}$} \)
        % matrix
        \( \lablnode{7}{2.5}{$\begin{bmatrix} 1 & 1 \\ 1 & 1 \end{bmatrix}$} \)

    \end{tikzpicture}
\end{center}

Next consider an arbitrary value $A(k)_{i,j}$ for any $k \geq 2$. This value is $1$ if the $k \times 1$ column with index $(i,j)$ can be part of a polygon mosaic, and $0$ otherwise. We can determine that specific values of $A(k+1)$ are multiples of $A(k)_{i,j}$ by considering the following operation on an arbitrary column with index $(i,j)$. Copy the column four times and replace the top two $0$'s of each column with the bottom row of labels from one of the four cell labelings below. 

\begin{center}
    \begin{tikzpicture}
        % row1
        \cell{0}{0}{1}{1}
        \cell{2}{0}{3}{1}
        \cell{4}{0}{5}{1}
        \cell{6}{0}{7}{1}

        \( \lablvertex{0}{0}{$0$} \)
        \( \lablvertex{1}{0}{$0$} \)
        \( \lablvertex{2}{0}{$0$} \)
        \( \lablvertex{3}{0}{$1$} \)
        \( \lablvertex{4}{0}{$1$} \)
        \( \lablvertex{5}{0}{$0$} \)
        \( \lablvertex{6}{0}{$1$} \)
        \( \lablvertex{7}{0}{$1$} \)

        \( \lablvertex{0}{1}{$0$} \)
        \( \lablvertex{1}{1}{$0$} \)
        \( \lablvertex{2}{1}{$0$} \)
        \( \lablvertex{3}{1}{$0$} \)
        \( \lablvertex{4}{1}{$0$} \)
        \( \lablvertex{5}{1}{$0$} \)
        \( \lablvertex{6}{1}{$0$} \)
        \( \lablvertex{7}{1}{$0$} \)
        
    \end{tikzpicture}
\end{center}

For example, for the $m=2$ column with index $(0,1)$, this operation looks like the following.

\begin{center}
    \begin{tikzpicture}
        % row1
        \cell{-4.5}{2.5}{-3.5}{3.5}
        \cell{-4.5}{3.5}{-3.5}{4.5}

        % row 1
        \( \lablvertex{-4.5}{2.5}{$0$} \)
        \( \lablvertex{-4.5}{3.5}{$0$} \)
        \( \lablvertex{-4.5}{4.5}{$0$} \)
        
        \( \lablvertex{-3.5}{2.5}{$0$} \)
        \( \lablvertex{-3.5}{3.5}{$1$} \)
        \( \lablvertex{-3.5}{4.5}{$0$} \)

        % label
        \( \lablnode{-4}{1.5}{$A(2)_{0,1}$} \)

        % arrow
        \( \lablnode{-2}{3.5}{$\pmb{\to}$} \)

        % four columns
        \cell{0}{0}{1}{1}
        \cell{0}{1}{1}{2}
        \cell{0}{2}{1}{3}

        \cell{2}{0}{3}{1}
        \cell{2}{1}{3}{2}
        \cell{2}{2}{3}{3}

        \( \lablvertex{0}{0}{$0$} \)
        \( \lablvertex{0}{1}{$0$} \)
        \( \lablvertex{0}{2}{$1$} \)
        \( \lablvertex{0}{3}{$0$} \)
        
        \( \lablvertex{1}{0}{$0$} \)
        \( \lablvertex{1}{1}{$1$} \)
        \( \lablvertex{1}{2}{$0$} \)
        \( \lablvertex{1}{3}{$0$} \)
        
        \( \lablvertex{2}{0}{$0$} \)
        \( \lablvertex{2}{1}{$0$} \)
        \( \lablvertex{2}{2}{$1$} \)
        \( \lablvertex{2}{3}{$0$} \)
        
        \( \lablvertex{3}{0}{$0$} \)
        \( \lablvertex{3}{1}{$1$} \)
        \( \lablvertex{3}{2}{$1$} \)
        \( \lablvertex{3}{3}{$0$} \)
        
        \cell{0}{4}{1}{5}
        \cell{0}{5}{1}{6}
        \cell{0}{6}{1}{7}
        
        \cell{2}{4}{3}{5}
        \cell{2}{5}{3}{6}
        \cell{2}{6}{3}{7}

        \( \lablvertex{0}{4}{$0$} \)
        \( \lablvertex{0}{5}{$0$} \)
        \( \lablvertex{0}{6}{$0$} \)
        \( \lablvertex{0}{7}{$0$} \)
        
        \( \lablvertex{1}{4}{$0$} \)
        \( \lablvertex{1}{5}{$1$} \)
        \( \lablvertex{1}{6}{$0$} \)
        \( \lablvertex{1}{7}{$0$} \)
        
        \( \lablvertex{2}{4}{$0$} \)
        \( \lablvertex{2}{5}{$0$} \)
        \( \lablvertex{2}{6}{$0$} \)
        \( \lablvertex{2}{7}{$0$} \)
        
        \( \lablvertex{3}{4}{$0$} \)
        \( \lablvertex{3}{5}{$1$} \)
        \( \lablvertex{3}{6}{$1$} \)
        \( \lablvertex{3}{7}{$0$} \)

        \( \lablnode{5}{3.5}{$\pmb{\to}$} \)

        \( \lablnode{7.5}{3.5}{$\begin{bmatrix} A(3)_{0,2} & A(3)_{0,3} \\ A(3)_{1,2} & A(3)_{1,3} \end{bmatrix}$} \)

    \end{tikzpicture}
\end{center}

This operation results in $4$ new columns that are represented in $A(k+1)$. In our example, specifically we get the following. 

$$
\begin{bmatrix} 
    A(3)_{0,2} & A(3)_{0,3} \\ 
    A(3)_{1,2} & A(3)_{1,3} 
\end{bmatrix} = 
\begin{bmatrix} 
    1A(2)_{i,j} & 1A(2)_{i,j} \\ 
    0A(2)_{i,j} & 1A(2)_{i,j} 
\end{bmatrix},
$$

Critically, this transformation \textit{only} changes the identity of the top two tiles. This implies that the same value coefficients computed by comparing $A(2)$ and $A(3)$ can be used for any $m \times 1$ column, as long as both column indices $(i,j)$ are congruent $\text{mod } 2$. Furthermore, if one writes $A(k)$ as the block matrix

$$A(k) = \begin{bmatrix} A_{0,0} & A_{0,1} \\ A_{1,0} & A_{1,1} \end{bmatrix},$$

where $A_{\hat{i},\hat{j}} \in \mathbb{R}^{2^{k-2} \times 2^{k-2}}$, then all column's represented in $A_{\hat{i},\hat{j}}$ have indices $(i,j) \equiv (\hat{i},\hat{j}) \text{ mod } 2$. This allows us to write that in general, if $A(k) = \begin{bmatrix} A_{0,0} & A_{0,1} \\ A_{1,0} & A_{1,1} \end{bmatrix},$
then

\begin{equation}\label{eqn: coefficient matrix prod}
    \begin{bmatrix}
        V_{0,0}A_{0,0} & V_{0,1}A_{0,0} & V_{0,2}A_{0,1} & V_{0,3}A_{0,1} \\
        V_{1,0}A_{0,0} & V_{1,1}A_{0,0} & V_{1,2}A_{0,1} & V_{1,3}A_{0,1} \\
        V_{2,0}A_{1,0} & V_{2,1}A_{1,0} & V_{2,2}A_{1,1} & V_{2,3}A_{1,1} \\
        V_{3,0}A_{1,0} & V_{3,1}A_{1,0} & V_{3,2}A_{1,1} & V_{3,3}A_{1,1} \\
    \end{bmatrix}.
\end{equation}


where $V \in \mathbb{R}^{4 \times 4}$ can be found after directly computing $A(2)$ and $A(3)$, then solving the following equation.

$$
\begin{bmatrix}
    A(3)_{0,0} & A(3)_{0,1} & A(3)_{0,2} & A(3)_{0,3} \\
    A(3)_{1,0} & A(3)_{1,1} & A(3)_{1,2} & A(3)_{1,3} \\
    A(3)_{2,0} & A(3)_{2,1} & A(3)_{2,2} & A(3)_{2,3} \\
    A(3)_{3,0} & A(3)_{3,1} & A(3)_{3,2} & A(3)_{3,3} \\
\end{bmatrix} = 
\begin{bmatrix}
    V_{0,0}A(2)_{0,0} & V_{0,1}A(2)_{0,0} & V_{0,2}A(2)_{0,1} & V_{0,3}A(2)_{0,1} \\
    V_{1,0}A(2)_{0,0} & V_{1,1}A(2)_{0,0} & V_{1,2}A(2)_{0,1} & V_{1,3}A(2)_{0,1} \\
    V_{2,0}A(2)_{1,0} & V_{2,1}A(2)_{1,0} & V_{2,2}A(2)_{1,1} & V_{2,3}A(2)_{1,1} \\
    V_{3,0}A(2)_{1,0} & V_{3,1}A(2)_{1,0} & V_{3,2}A(2)_{1,1} & V_{3,3}A(2)_{1,1} \\
\end{bmatrix}
$$

This is solved by

$$V = 
\begin{bmatrix} 
    1 & 1 & 1 & 1 \\ 
    1 & 1 & 0 & 1 \\ 
    1 & 0 & 1 & 1 \\ 
    1 & 1 & 1 & 1 
\end{bmatrix}
$$


which completes the proof.

\end{proof}

The method detailed in Theorem \ref{thm: main theorem} generalizes to other tile sets, which we tabularize in Section \ref{section: summary of results} without proof. Interestingly, the method not only generalizes to other tile sets, but can also be augmented to enumerate the more complicated ``messy" polygon mosaics.



As demonstrated in the proof of Theorem \ref{thm: messy mosaics}, the enumeration of both polygon mosaics and mosaics that do not contain a polygon share the same structure, and only differ by the identity of the matrices $A(2), V$. We summarize these matrices for various collections of tiles below.

\begin{center}
    \begin{tblr}{
  colspec = {X[c,h]X[c]X[c]X[c]},
  stretch = 0,
  rowsep = 6pt,
  hlines = {black, 1pt},
  vlines = {black, 1pt},
}
  \textbf{Tile Set} & \textbf{Polygon Mosaics} & \textbf{Messy Polygon Mosaics}\\
  
  \resizebox{0.3\textwidth}{!}{%
  \begin{tikzpicture}
      \cellA{0}{0}{1}{1}
      \cellB{2}{0}{3}{1}
      \cellC{0}{2}{1}{3}
      \cellD{2}{2}{3}{3}
      
      \cellE{4}{0}{5}{1}
      \cellF{4}{2}{5}{3}

      \cell{6}{1}{7}{2}

    \end{tikzpicture}
    } 
    & 
    $\begin{bmatrix}
    1 & 1 \\
    1 & 1
    \end{bmatrix},
    \begin{bmatrix} 
    1 & 1 & 1 & 1 \\ 
    1 & 1 & 0 & 1 \\ 
    1 & 0 & 1 & 1 \\ 
    1 & 1 & 1 & 1 
    \end{bmatrix}$
    & 
    $\begin{bmatrix}
        7^2 & 1 \\
        -1 & 1
    \end{bmatrix},
    \begin{bmatrix}
        7 & \frac{1}{7} & 7 & 1 \\
        -\frac{1}{7} & 1 & 0 & 1 \\
        7 & 0 & 7  & 1 \\
        1 & -1 & 1 & 7 \\
    \end{bmatrix}$
    \\
    
    \resizebox{0.3\textwidth}{!}{%
    \begin{tikzpicture}
        \cellA{0}{0}{1}{1}
        \cellB{2}{0}{3}{1}
        \cellC{0}{2}{1}{3}
        \cellD{2}{2}{3}{3}
        
        \cellE{4}{0}{5}{1}
        \cellF{4}{2}{5}{3}
        % invisible cell for spacing
        \spacecell{6}{1}{7}{2}
    \end{tikzpicture}
    } 
    & 
    TODO 
    & 
    $\begin{bmatrix}
        6^2 & 1 \\
        -1 & 1
    \end{bmatrix},
    \begin{bmatrix}
        6 & \frac{1}{6} & 6 & 1 \\
        -\frac{1}{6} & 1 & 0 & 1 \\
        6 & 0 & 6  & 1 \\
        1 & -1 & 1 & 6 \\
    \end{bmatrix}$ 
    \\

    \resizebox{0.3\textwidth}{!}{%
    \begin{tikzpicture}
        \cellA{0}{0}{1}{1}
        \cellB{2}{0}{3}{1}
        \cellC{0}{2}{1}{3}
        \cellD{2}{2}{3}{3}
        
        \cell{4}{1}{5}{2}
        % invisible cell for spacing
        \spacecell{6}{1}{7}{2}
        
    \end{tikzpicture}
    } 
    & 
    $\begin{bmatrix}
        1 & 1 \\
        1 & 0
    \end{bmatrix}$, TODO 
    & 
    TODO
    \\

    \resizebox{0.3\textwidth}{!}{%
    \begin{tikzpicture}
        \cellA{0}{0}{1}{1}
        \cellB{2}{0}{3}{1}
        \cellC{0}{2}{1}{3}
        \cellD{2}{2}{3}{3}
        % invisible cell for spacing
        \spacecell{4}{1}{5}{2}
        % invisible cell for spacing
        \spacecell{6}{1}{7}{2}
    \end{tikzpicture}
      } 
    & 
    TODO
    & 
    TODO 
    \\
  
\end{tblr}
\end{center}


\begin{theorem}
    \label{thm: growth rate}    
    Let the probability that an $(m,n)$-mosaic \textit{does not} contain a SAP be denoted $p_{m,n} = \frac{t_{m,n}}{7^{nm}}$. Then the growth rate of the main diagonal $p_{n,n}$ has
    $$\gamma = \lim_{n \to \infty} \frac{p_{n+1,n+1}p_{n-1,n-1}}{p_{n,n}^2} = ?$$
\end{theorem}
